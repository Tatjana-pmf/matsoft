\documentclass{beamer}
\usetheme{CambridgeUS}

\usepackage{tikz}

\usepackage[croatian]{babel}
\usepackage[autostyle]{csquotes}
\usepackage[utf8]{inputenc}
\usepackage[T1]{fontenc}
\usepackage{amsmath}
\usepackage{url}
\usepackage{fancybox}
\usepackage{wasysym}
\usepackage{listings}
\usepackage{multirow}

\newcommand{\A}[1]{#1&\texttt{\string#1}\hspace*{1ex}}

\usetikzlibrary{arrows,shapes,positioning}

\hypersetup{plainpages=false,bookmarksopen,linkcolor=violet,%
pdfpagemode=FullScreen,colorlinks=true%
}
\hypersetup{pdftitle={Uvod i osnove}}
\hypersetup{baseurl={web.math.pmf.unizg.hr/nastava/matsoft/}}
\hypersetup{pdfsubject={LaTeX}}
\hypersetup{pdfauthor={Ivica Nakić}}

\lstset{language=[LaTeX]TeX,extendedchars=true,backgroundcolor=\color{gray!20},identifierstyle=,stringstyle=\rmfamily,%
columns=flexible,showstringspaces=false,keywordstyle=\color{violet}\bfseries}

\lstset{literate=%
{Š}{{\v{S}}}1
{Đ}{{\Dj{}}}1
{Č}{{\v{C}}}1
{Ć}{{\'C}}1
{Ž}{{\v{Z}}}1
{š}{{\v{s}}}1
{Đ}{{\dj{}}}1
{č}{{\v{c}}}1
{ć}{{\'c}}1
{ž}{{\v{z}}}1
}


\title{\LaTeX{} --- Uvod i osnove}
\subtitle{}
\author{Ivica Nakić \\ \texttt{nakic@math.hr}}
\institute[PMF--MO]{Matematički odsjek Prirodoslovno--matematičkog fakulteta}
\date[2013/14]{Matematički software, 2014/15}


% naredbe koje se koriste u prezentaciji
\newcommand{\vek}{(x_1,\ldots,x_n)}
\newcommand{\veki}[1]{(#1_1,\ldots,#1_n)}
\newcommand{\vekii}[2]{(#1_1,\ldots,#1_#2)}
\newcommand{\vazno}[1]{\textbf{#1}}
\DeclareMathOperator{\tg}{tg}
\DeclareMathOperator*{\esup}{ess\;sup}
\newcommand{\mvek}{\ensuremath{(x_1,\ldots,x_n)}}

% okoline za teoreme koje se koriste u prezentaciji
\newtheorem{thm}{Teorem}
\newtheorem*{rol}{Rolleov teorem}
\newtheorem{cor}[thm]{Korolar}
\theoremstyle{remark}
\newtheorem{slutnja}{Slutnja}
\theoremstyle{definition}
\newtheorem{dfn}[slutnja]{Definicija}

\begin{document}

\begin{frame}
  \maketitle  
\end{frame}

\begin{frame}
\frametitle{Pregled}
  \tableofcontents  
\end{frame}

\section{Uvod}

\begin{frame}
\frametitle{Što je \LaTeX{}?}
\LaTeX{} je slovoslagarski program, nastao kao proširenje programa \TeX{} kojeg je napisao Donald Knuth.

Što je slovoslagarski program?

Proces pripravljanja dokumenta na računalu se sastoji od četiri faze:

\begin{itemize}
\item tekst se unosi u računalo
\item unešeni tekst se formatira u retke, paragrafe i stranice
\item izlazni tekst se prikazuje na naslonu računala
\item dokument se ispisuje
\end{itemize}

U većini programa za procesiranje teksta (eng.~\textit{word processors}) te četiri faze su integrirane. Ali \TeX{} služi samo za izvršavanje druge faze.
\end{frame}

\begin{frame}[fragile]
\frametitle{Mali primjer}
\LaTeX{} u akciji!

Jedan (vrlo) kratak dokument formatiran u \LaTeX u.

\begin{lstlisting}
\documentclass{article}
\begin{document}
Ovo je moj \emph{prvi} dokument 
u \LaTeX u.
\end{document}
\end{lstlisting}
\end{frame}


\begin{frame}
\frametitle{Zašto \LaTeX{}?}
Ali čemu toliki trud? Zašto jednostavno ne koristiti neki \emph{word processor}?

Odgovor je dao sam autor D.~Knuth: svrha \TeX a (a onda i \LaTeX a) je omogućiti kreiranje \textcolor{red}{lijepih} dokumenata, naročito onih koji sadrže \textcolor{red}{puno Matematike}. 

Vrlo je teško, katkada nemoguće, producirati kompleksne matematičke formule pomoću programa za procesiranje teksta. Ali čak i u slučaju običnog teksta, ukoliko želite da vaš dokument izgleda \emph{stvarno lijepo}, \LaTeX{} je prirodan odabir.
\end{frame}

\begin{frame}[<+->]
\frametitle{Važni linkovi}
\begin{itemize}
      \item \url{http://www.latex-project.org/}
      \item \url{http://www.tug.org/texlive/}
      \item \url{http://www.ctan.org/}
    \end{itemize}    

\end{frame}

\begin{frame}
\frametitle{Ostali linkovi}
\begin{itemize}
      \item \url{http://miktex.org/}
      \item \url{http://texstudio.sourceforge.net/}
      \item \url{http://www.xm1math.net/texmaker/}
      \item \url{http://www.lyx.org/} 
      \item \url{http://www.texniccenter.org/}
      \item \url{https://www.sharelatex.com/}
      \item \url{https://www.writelatex.com/}
      \item \url{http://fiduswriter.org/}
      \item \url{http://tex.stackexchange.com/}
      \item \url{http://www.texample.net/}
      \item \url{http://www.latextemplates.com/}
      \item \url{http://www.tug.org/interest.html}
      \item \url{http://www.stdout.org/~winston/latex/latexsheet-a4.pdf}
      \item \url{webdemo.visionobjects.com/equation.html?locale=default}
    \end{itemize}    

\end{frame}


\section{Jednostavno slovoslagarstvo}


\begin{frame}
\frametitle{Praznine}
\begin{itemize}
\item Paragrafi se odvajaju praznim retkom. Prva riječ u paragrafu je malo uvučena. Ukoliko to želimo spriječiti koristimo naredbu \textcolor{violet}{\textbackslash noindent}. 
\item Prelazak u novi red u editoru \textcolor{red}{ne znači} i prelazak u novi red u dokumentu. Razmaci se također ignoriraju u \LaTeX u. U novi red (ukoliko odluku ne želimo prepustiti \LaTeX u) možemo preći pomoću \textcolor{violet}{\textbackslash \textbackslash} ili \textcolor{violet}{\textbackslash newline}.
\item Prelazak na novu stranicu (ukoliko odluku ne želimo prepustiti \LaTeX u) se postiže pomoću naredbe \textcolor{violet}{\textbackslash newpage}.
\item Točka koja ne dolazi poslije velikog slova označava kraj rečenice. To pravilo možemo promijeniti pomoću naredbi \textcolor{violet}{\textbackslash @} i \textcolor{violet}{\textbackslash\textvisiblespace} \\
(primjer: \texttt{npr.\textbackslash\textvisiblespace nije kraj}).
\end{itemize}
\end{frame}


\begin{frame}
\frametitle{Naše stvari}
\begin{itemize}
\item Hrvatske inačice navodnika: \glqq navodnici'' i \frqq navodnici\flqq{} se dobijaju na sljedeći način:
\begin{center}
\textcolor{violet}{\textbackslash glqq navodnici''} \\
i\\
\textcolor{violet}{\textbackslash frqq navodnici \textbackslash flqq\{\}}
\end{center}
Ovdje \textcolor{violet}{''} označavaju dva jednostruka navodnika. 
Drugi način je korištenje paketa \textcolor{violet}{csquotes} s opcijom \texttt{croatian}. Tada pišemo \textcolor{violet}{\textbackslash enquote\{tekst\}} da bi dobili \enquote{tekst}.
\item Navodnici u engleskom pismu: ``quotes'' su dobijeni pomoću dva jednostruka navodnika \textcolor{violet}{`} i dva jednostruka navodnika \textcolor{violet}{'} 
\item Hrvatski dijakritički znakovi se mogu unositi direktno preko tipkovnice ako podesimo kodiranje, ili pomoću naredbi \textcolor{violet}{\textbackslash 'c} za slovo ć, \textcolor{violet}{\textbackslash  v c} za slovo č, \textcolor{violet}{\textbackslash v ž} za slovo ž, \textcolor{violet}{\textbackslash v s} za slovo š, te \textcolor{violet}{\textbackslash dj} za slovo đ. I dijakritičke znakove iz ostalih pisama možemo slično unositi npr. Schr\"odinger. 
\end{itemize}
\end{frame}


\begin{frame}
\frametitle{Crtice \& specijalni znakovi}
Crtice: postoje tri vrste \textcolor{violet}{-, --, ---}. Prva služi za rastavljanje slogova (hiphenaciju), druga za slijedove, treća za komentare. 

\textcolor{green}{X-zrake se diskutiraju na stranicama 221--225 treće knjige --- posvećene elektromagnetskim valovima.}

Unos: \textcolor{violet}{-}, \textcolor{violet}{-\ -}, \textcolor{violet}{-\ -\ -}.

\pause

Postoji deset specijalnih znakova koji su dijelovi \LaTeX{} naredbi i koje moramo drugačije unositi. To su:

\begin{center}
\begin{small}
  \begin{tabular}{|cl|cl|}
    \hline
    \rule{0pt}{12pt}\A{\textasciitilde}  & \A{\&}\\ 
    \A{\#}               & \A{\_}             \\
    \A{\$}               & \A{\textbackslash} \\
    \A{\%}               & \A{\{}             \\
    \A{\textasciicircum} & \A{\}}             \\[3pt]
    \hline
  \end{tabular}
\end{small}
\end{center}

\pause

Komentari se unose tako da se stave između dva znaka \%. Komentari se mogu protezati u više redaka.
\end{frame}


\begin{frame}
\frametitle{Pozicioniranje teksta}
\tikzstyle{na} = [baseline=-.5ex]
\begin{flushleft}
\tikz[na,remember picture]{ 
\node[fill=blue!20,ellipse,anchor=base] (a) {4. travnja 2012.};
}
\end{flushleft}
  \begin{center}
    \tikz[na,remember picture]{ 
    \node[fill=blue!20,ellipse,anchor=base] (b) {\TeX nički Institut};
    }
\\          
          Certifikat
  \end{center}

  \noindent Potvrđuje se da je Pero Perić uspješno pohađao kurs na ovom Institutu i da je 
certificiran \TeX ničar.
  \begin{flushright}
    \tikz[na,remember picture]{ 
    \node[fill=blue!20,ellipse,anchor=base] (c) {Direktor \TeX ničkog Instituta};
    }
  \end{flushright}
\onslide<2->{
Naredbe, tj. okoline su \textcolor{violet}{\textbackslash begin\{\smiley\} ... \textbackslash end\{\smiley\}} gdje je \\
\textcolor{violet}{\smiley} $=$
\tikz[baseline,remember picture] \node[anchor=base] (d) {\textcolor{violet}{flushleft},};
\tikz[baseline,remember picture] \node[anchor=base] (f) {\textcolor{violet}{center},};
\tikz[baseline,remember picture] \node[anchor=base] (e) {\textcolor{violet}{flushright}.};
}

\begin{tikzpicture}[remember picture,overlay]
  \path[->,red,very thick]<2-> (a) edge [bend right] (d);
  \path[->,red,very thick]<2-> (b) edge (f);
  \path[->,red,very thick]<2-> (c) edge [bend left] (e);
\end{tikzpicture}
\end{frame}

\begin{frame}
\frametitle{Fontovi}
\setbeamertemplate{itemize item}{\RIGHTarrow}
\begin{itemize}
\item \textmd{\textbackslash textmd medium}
\item \textbf{\textbackslash textbf boldface}
\item \textit{\textbackslash textit italic}
\item \textup{\textbackslash textup upright}
\item \textsl{\textbackslash textsl slanted} 
\item \textsc{\textbackslash textsc small cap}
\item \textrm{\textbackslash textrm roman}
\item \textsf{\textbackslash textsf sans erif}
\item \texttt{\textbackslash texttt typewriter}
\end{itemize}

\end{frame}

\begin{frame}[fragile]
\frametitle{Fontovi}
Naravno, možemo mijenjati i fontove, što se najlakše može raditi s \mbox{X\lower0.5ex\hbox{\kern-0.15em\reflectbox{E}}\kern-0.1em\LaTeX}om ili 
Lua\LaTeX{}om. Pri kompajliranju tada treba koristiti odgovarajući program \texttt{xelatex} odnosno \texttt{lualatex}.   

\begin{lstlisting}
\documentclass[12pt]{article}
\usepackage[croatian]{babel}
\usepackage{xunicode}
\usepackage{xltxtra}
\usepackage{fontspec}
\defaultfontfeatures{Mapping=tex-text}
\setmainfont[Ligatures={Common}]{Hoefler Text}  
\end{lstlisting}

\end{frame}

\begin{frame}
\frametitle{Veličina slova}
\begin{itemize}
\item \tiny{\textbackslash tiny}
\item \scriptsize{\textbackslash scriptsize}
\item \footnotesize{\textbackslash footnotesize}
\item \small{\textbackslash small}
\item \normalsize{\textbackslash normalsize}
\item \large{\textbackslash large}
\item \Large{\textbackslash Large}
\item \LARGE{\textbackslash LARGE}
\item \huge{\textbackslash huge}
\item \Huge{\textbackslash Huge}
\end{itemize}
\end{frame}

\begin{frame}
\frametitle{Dekoracije i razmaci}
\begin{itemize}
\item \underline{\textbackslash underline}
\item \frame{\textbackslash frame}
\item \fbox{\textbackslash fbox}
\item \textbackslash raisebox \raisebox{1ex}{gore} ili \raisebox{-1ex}{dolje}
\item a možemo i ovo \shadowbox{shadowbox}
\item ili ovo \Ovalbox{Ovalbox}
\item ili ovo \rotatebox{-25}{\fbox{rotatebox}}
\end{itemize}
\setbeamertemplate{itemize item}[default]
Za kutije \textcolor{violet}{shadowbox} i \textcolor{violet}{Ovalbox} potrebno je uključiti paket \textcolor{violet}{fancybox}, dok je za kutiju \textcolor{violet}{rotatebox} potrebno uključiti paket \textcolor{violet}{graphicx}.

\pause

Razmaci među retcima: \textbackslash bigskip, \textbackslash medskip, \textbackslash smallskip.

\end{frame}

\section{Struktura dokumenta}

\begin{frame}[<+->]
\frametitle{Struktura dokumenta}
\begin{itemize}
\item \textcolor{orange}{Zaglavlje}
\begin{itemize}
\item \textbackslash documentclass[\textcolor{brown}{opcije}]\{\textcolor{brown}{klase}\}
\item \textcolor{brown}{opcije}: veličina fonta (\textcolor{brown}{10pt}, \textcolor{brown}{11pt}, \textcolor{brown}{12pt}), veličina papira(\textcolor{brown}{a4paper}, \textcolor{brown}{letterpaper}, \textcolor{brown}{a5paper}), jednostupčani dokument (\textcolor{brown}{onecolumn} --- podrazumijevana vrijednost), dvostupčani dokument (\textcolor{brown}{twocolumn}), jednostrano (\textcolor{brown}{oneside}), dvostrano (\textcolor{brown}{twoside}), ...
\item \textcolor{brown}{klase}: \textcolor{brown}{book}, \textcolor{brown}{report}, \textcolor{brown}{article}, \textcolor{brown}{letter}, ...
\item \textbackslash usepackage[\textcolor{brown}{opcije}]\{\textcolor{brown}{paket}\}
\item \textbackslash pagestyle\{\textcolor{brown}{opcije}\} --- opcije su npr. \textcolor{brown}{plain},\textcolor{brown}{empty},...
\end{itemize}
\item \textcolor{orange}{Tijelo}

\textbackslash begin\{document\} ...\textbackslash end\{document\}
\end{itemize}
\end{frame}

\begin{frame}[fragile]
\frametitle{Naslov}
Jednostavan primjer:
\begin{lstlisting}
\documentclass{article}
\usepackage[croatian]{babel}
\usepackage[utf8]{inputenc}
\usepackage[T1]{fontenc}
\title{Naslov}
\author{Ja}
\date{danas}

\begin{document}

\maketitle

Neke umotvorine\ldots

\end{document}
\end{lstlisting}
\end{frame}

\begin{frame}[fragile]
\frametitle{Naslov 2}
Kako postići da naslov bude na posebnoj stranici? Maknuli smo i naredbu \textcolor{violet}{\textbackslash date}.
\begin{lstlisting}
\documentclass[titlepage]{article}
\usepackage[croatian]{babel}
\usepackage[utf8]{inputenc}
\usepackage[T1]{fontenc}
\title{Naslov}
\author{Ja}
%\date{danas}

\begin{document}

\maketitle

Neke umotvorine\ldots

\end{document}
\end{lstlisting}
\end{frame}

\begin{frame}[fragile]
\frametitle{Autor(i)}
Što ukoliko ima više autora? Gdje dolaze podaci o adresi, web stranici, e-mailu?

Za sve to služi naredba \textcolor{violet}{\textbackslash author}. Primjer:
\begin{lstlisting}
\author{Autor 1 \\
Adresa\\
E-mail\\
\and
\Autor 2\\
Adresa\\
E-mail\\
Institucija}
\end{lstlisting}
\end{frame}

\begin{frame}[<+->]
\frametitle{Podjela dokumenta}
\begin{itemize}
\item \textbf{Sažetak} stavljamo poslije naslova. Okolina u kojoj pišemo sažetak je \textcolor{violet}{abstract}
\item Dokument obično dijelimo na dijelove, poglavlja, sekcije, podsekcije, ... Odgovarajuće naredbe u \LaTeX u su \textcolor{violet}{\textbackslash part}, \textcolor{violet}{\textbackslash chapter} --- (samo za klase \textcolor{violet}{book} i \textcolor{violet}{report}), \textcolor{violet}{\textbackslash section} i \textcolor{violet}{\textbackslash subsection}.

Primjer je npr. \textcolor{violet}{\textbackslash section\{Naslov sekcije\}}.
\item Dijelovi dokumenta će automatski biti numerirani. Ukoliko npr. želite nenumeriranu sekciju koristite 
\textcolor{violet}{\textbackslash section*}.  
\item Za još finiju podjelu postoje i \textcolor{violet}{\textbackslash paragraph} i 
\textcolor{violet}{\textbackslash subparagraph}.
\end{itemize}
\end{frame}

\begin{frame}
\frametitle{Sadržaj}

\LaTeX sam vodi brigu o sadržaju, te je dovoljno na odgovarajuće mjesto (gdje želimo staviti sadržaj) staviti naredbu \textcolor{violet}{\textbackslash tableofcontents}. 

Da bi smo bili sigurni da sadržaj odgovara trenutnom stanju dokumenta, trebamo dvaput \LaTeX irati dokument.

Poglavlja, sekcije, \ldots koje smo označili sa zvjezdicom ne ulaze u sadržaj. Ukoliko želimo upisati u sadržaj nešto što \LaTeX ne radi 
automatski (kao npr.\ \textcolor{violet}{\textbackslash chapter*\{Naslov\}}), odmah poslije ove naredbe stavimo (u slučaju gornjeg primjera):

\textcolor{violet}{\textbackslash addcontentsline\{toc\}\{chapter\}\{\textbackslash numberline\{\}Naslov\}}

\end{frame}

\begin{frame}[fragile]
\frametitle{Nabrajanje}
Numerirana lista se formatira pomoću okoline \textcolor{violet}{enumerate}. Svaki element list označavmo s naredbom \textcolor{violet}{\textbackslash item}. Primjer:
\begin{lstlisting}
\begin{enumerate}
\item Linearna algebra 1
\item Linearna algebra 2
\item Elementarna matematika 1
\item Elementarna matematika 2
\end{enumerate}
\end{lstlisting}
Naravno, numerirane liste možemo ugniježđavati. U prvom nivou numeracija je 1, 2, ..., u drugom (a), (b), (c), ..., u trećem i, ii, iii, ...
\end{frame}

\begin{frame}[fragile]
\frametitle{Liste}
Ukoliko ne želimo numerirati liste, koristimo okolinu \textcolor{violet}{itemize}. Sintaksa je ista kao i za numeriranu listu. 

Ukoliko želimo kreirati listu a--la indeks pojmova, koristimo okolinu \textcolor{violet}{description}. Primjer:
\begin{lstlisting}
\begin{description}
\item[slon] veliki sisavac\ldots
\item[Mac OS X] operativni sustav\ldots 
\item[nogomet] sport\ldots
\end{description}
\end{lstlisting}
U okolinama \textcolor{violet}{itemize} i \textcolor{violet}{enumerate} možemo promijeniti znak ispred elementa. 
\end{frame}

\begin{frame}[<+->]
\frametitle{Okoline za...}
\begin{itemize}
\item Ukoliko želimo u dokument staviti citat koristimo okolinu \textcolor{violet}{quote}.
\item Ukoliko želimo u dokument staviti duži citat (duži od jednog paragrafa) koristimo okolinu 
 \textcolor{violet}{quotation}.
\item Ako unosimo tekst pjesme, prava okolina za to je \textcolor{violet}{verse}. U toj okolini \LaTeX{} ne prelazi u novi red sam, nego mi sami moramo eksplicitno prijeći u novi red s dvije obrnute kose crte \textcolor{violet}{\textbackslash\textbackslash}. U novu strofu prelazimo pomoću jednog praznog retka.
\item Ako želite da se unešeni tekst prikaže točno onako kako je unesen (za unos npr. programskog k\^oda), koristimo okolinu \textcolor{violet}{lstlisting} iz paketa \texttt{listings}. Ako još usto želimo da se praznine u tekstu fizički obilježe, potrebno je koristiti okolinu \textcolor{violet}{lstlisting*}. 

Druga opcija je paket \texttt{minted}.
\end{itemize}
\end{frame}


\begin{frame}
\frametitle{Fusnote}
Unos fusnota je vrlo jednostavan\footnote{Ova fusnota je unešena na sljedeći način:\\
...jednostavan\textbackslash footnote\{Ova fusnota je...\}}

Fusnotu stavljamo odmah poslije riječi na koju se odnosi.\footnote{
Ukoliko se fusnota odnosi na rečenicu ili paragraf, fusnotu stavljamo odmah poslije točke. }
\end{frame}

\begin{frame}[fragile]
\frametitle{Tablice}
U \LaTeX u tablice kreiramo pomoću okoline \textcolor{violet}{tabular}. Jedan primjer tablice:
\begin{lstlisting}
\begin{tabular}{ll}
\textbf{Ime} & \textbf{Adresa} \\
Pero Perić & Kozji put 16B \\
Ivo Ivić &  Pod lipom bb \\
Mare Marić & Zelena obala 3 \\
\end{tabular}
\end{lstlisting}
Znak \textcolor{violet}{\&} služi da odijelljivanje elemenata u retku. Naravno, \textcolor{violet}{\textbackslash\textbackslash} služi za prelazak u novi red.

Opcije \textcolor{violet}{ll} označavaju pozicioniranje teksta u stupcima. Oznake:
\begin{tabular}{ll}
\textcolor{violet}{l} & lijevo \\
\textcolor{violet}{c} & centrirano \\
\textcolor{violet}{r} & desno 
\end{tabular}
\end{frame}

\begin{frame}[fragile]
\frametitle{Tablice 2}
Možemo odijeliti stupce i retke linijama:
\begin{lstlisting}
\begin{tabular}{|c|c|} \hline
\textbf{Ime} & \textbf{Adresa} \\ \hline
Pero Perić & Kozji put 16B \\ \hline
Ivo Ivić &  Pod lipom bb \\ \hline
Mare Marić & Zelena obala 3 \\ \hline
\end{tabular}
\end{lstlisting}

Rezultat:
\begin{tabular}{|c|c|} \hline
\textbf{Ime} & \textbf{Adresa} \\ \hline
Pero Perić & Kozji put 16B \\ \hline
Ivo Ivić &  Pod lipom bb \\ \hline
Mare Marić & Zelena obala 3 \\ \hline
\end{tabular}

\textcolor{violet}{\textbackslash hline} nam daje vodoravne linije, a okomite linije smo dobili zbog \textcolor{violet}{\{|c|c|\}}. Naravno \textcolor{violet}{c}-ovima smo centrirali imena i adrese. 
\end{frame}

\begin{frame}
\frametitle{Tablice 3}
U \LaTeX u se mogu kreirati jednstavno i kompliciranije tablice, kao što je ova:
 \renewcommand{\multirowsetup}{\centering}
\begin{center}
\begin{tabular}{|l|r|r|}
    \hline
    \multirow{3}{1.5cm}{Planet}   
            & \multicolumn{2}{p{3.5cm}|}%
                 {\centering Udaljenost od sunca \\ (milijuni km)}\\
    \cline{2-3}
            & \multicolumn{1}{c|}{Maksimum}    
                     & \multicolumn{1}{c|}{Minimum}\\
    \hline
    Merkur & 69.4   & 46.8\\
    Venera   & 109.0  & 107.6\\
    Zemlja   & 152.6  & 147.4\\
    Mars    & 249.2  & 207.3\\
    Jupiter & 817.4  & 741.6\\
    Saturn  & 1512.0 & 1346.0\\
    Uran  & 3011.0 & 2740.0\\
    Neptun & 4543.0 & 4466.0\\
    \hline
\end{tabular}
\end{center}
\end{frame}

\begin{frame}
\frametitle{Tablice 4}
Prethodna tablica je kreirana koristeći paket \textcolor{violet}{multirow}. 

Postoje još mnogi drugi paketi koji olakšavaju kreiranje kompleksnih tablica: \textcolor{violet}{longtable}, 
 \textcolor{violet}{tabularx},  \textcolor{violet}{dcolumn},  \textcolor{violet}{delarray},  \textcolor{violet}{hhline},...

Postoje još neke standardne okoline za kreiranje tabličnih podataka, kao što je  \textcolor{violet}{tabbing}. Ali njima se nećemo baviti. 

\LaTeX{} tablice se mogu generirati i online, pomoću web stranice \url{http://www.tablesgenerator.com/}.

\end{frame}

\section{Klase dokumenata}

\begin{frame}[fragile]
\frametitle{Pisanje pisama}
Kao primjer nestandardne klase dokumenata, pokazat ćemo kako se koristi klasa za formatiranje pisama  \textcolor{violet}{letter}. Evo jednog tipičnog primjera:
\begin{lstlisting}
\documentclass[a4paper,12pt]{letter}
\usepackage[croatian]{babel}
\usepackage[utf8]{inputenc}
\usepackage[T1]{fontenc}
\begin{document}
\begin{letter}{Dr. Strogi Nastavnik\\
         PMF--MO\\
         Bijenička cesta 30\\
         10 000 Zagreb}
\address{S.C. Cvjetno naselje\\
             Soba 125/3\\
       Zagreb}
\end{lstlisting}
\end{frame}

\begin{frame}[fragile]
\frametitle{Nastavak pisma}
\begin{lstlisting}
\opening{Poštovani,}
molim Vas da mi dodijelite drugi potpis 
iz kolegija Računarski praktikum 3.
\signature{Ivica Mali\\
                Student}
\closing{Sa štovanjem,}
\encl{Indeks}
\end{letter}
\end{document}
\end{lstlisting}
\end{frame}

\begin{frame}
\frametitle{Ostale zanimljive klase}
\begin{center}
  Beamer, memoir, todonotes, \ldots.
\end{center}    

\end{frame}

\section{Naredbe}

\begin{frame}[fragile]
\frametitle{Naredbe}
U  \LaTeX u možemo definirati nove naredbe koristeći naredbu \textcolor{violet}{\textbackslash newcommand}. Osnovna sintaksa je 
\begin{lstlisting}
\newcommand{ime_naredbe}{kod} 
\end{lstlisting}

Primjer je 

\begin{lstlisting}
\newcommand{\vek}{(x_1,\ldots,x_n)}
\end{lstlisting}

Sada k\^od \$\textbackslash vek\$ daje: $\vek$.

Naravno, isti efekt se lako postigne i definiranjem makroa u editoru, što je katkad i transparentnije rješenje. Naredba može imati i ulazne parametre. Sintaksa je 
\begin{lstlisting}
\newcommand[n]{ime_naredbe}{kod} 
\end{lstlisting}
gdje je n broj parametara koje u k\^odu pozivamo pomoću \#1,...,\#n.
\end{frame}

\begin{frame}[fragile]
\frametitle{Naredbe 2}
Primjeri:
\begin{small}
\begin{lstlisting}
\newcommand{\veki}[1]{(#1_1,\ldots,#1_n)}
\newcommand{\vekii}[2]{(#1_1,\ldots,#1_#2)}
\end{lstlisting}
\end{small}
Kako ih koristimo? Npr. 
\begin{lstlisting}
\[\veki{\alpha}=\vekii{\beta}{m}\]
\end{lstlisting}
nam daje

\[\veki{\alpha}=\vekii{\beta}{m}\]

Koji put je zgodno definirati novu naredbu da bismo logički označili tekst, npr. 
\begin{lstlisting}
\newcommand{\vazno}[1]{\textbf{#1}}
\end{lstlisting}
nam omogućava da pišemo
\begin{lstlisting}
\vazno{Ovo je važno}
\end{lstlisting}
da bi smo dobili:
\vazno{Ovo je važno}
\end{frame}

\begin{frame}[fragile]
\frametitle{Naredbe 3}
U slučaju da želimo definirati niovi matematički operator, na raspolaganju nam je naredba \textcolor{violet}{\textbackslash DeclareMathOperator}. Npr.
\begin{lstlisting}
\DeclareMathOperator{\tg}{tg}
\end{lstlisting}
nam omogućava da pišemo \textcolor{violet}{\$ \textbackslash tg x=\textbackslash sin x / \textbackslash cos x\$}: $ \tg x=\sin x / \cos x$. Slično
\begin{lstlisting}
\[ \|f\|_{\infty}=
\esup_{x\in\mathbb{R}} |f(x)| \]
\end{lstlisting}
nam daje
\[ \|f\|_{\infty}=
\esup_{x\in\mathbb{R}} |f(x)| \]
ukoliko smo u zaglavlje stavili
\begin{lstlisting}
\DeclareMathOperator*{\esup}{ess\;sup}
\end{lstlisting}
\end{frame}

\begin{frame}[fragile]
\frametitle{Naredbe 4}
Nove naredbe možemo definirati bilo gdje u dokumentu, ali je \textcolor{red}{dobra} konvencija da to napravimo u zaglavlju.

Ukoliko napišemo \textcolor{violet}{\textbackslash vek} izvan matematičkog teksta, pri prevođenju će nam biti javljena greška. Ali ukoliko modificiramo malo našu naredbu tako da glasi:
\begin{lstlisting}
\newcommand{\vek}
{\ensuremath{(x_1,\ldots,x_n)}}
\end{lstlisting}
onda možemo pisati i \textcolor{violet}{\$\textbackslash vek\$} i \textcolor{violet}{\textbackslash vek}.

Postoji i naredbe \textcolor{violet}{\textbackslash operatorname} i \textcolor{violet}{\textbackslash operatorname*} 
koje omogućavaju definiranje binarnih operatora.

Također postoji i naredba \textcolor{violet}{\textbackslash mathop} koja je analogon naredbama \textcolor{violet}{\textbackslash mathrel} i \textcolor{violet}{\textbackslash mathbin}.
\end{frame}

\section{Teoremi i slične okoline} % (fold)
\label{sec:teoremi}

\begin{frame}[fragile]
\frametitle{\textcolor{violet}{\textbackslash newtheorem}}
Posredstvom naredbe \textcolor{violet}{\textbackslash newtheorem} iz paketa \textcolor{violet}{amsthm} omogućeno je kreiranje okolina za unos teorema, propozicija i sličnih konstrukata. Osnovna sintaksa  je 
\begin{lstlisting}
\newtheorem{ime_okoline}{ime_konstrukta}
\end{lstlisting}
Na primjer:
\begin{lstlisting}
\newtheorem{thm}{Teorem}
\end{lstlisting}
omogućava unos teorema na sljedeći način:
\begin{lstlisting}
\begin{thm}
Evo jednog teorema.
\end{thm}
\end{lstlisting}
Rezultat je:
{\color{brown}
\begin{thm}
Evo jednog teorema.
\end{thm}
}
\end{frame}

\begin{frame}[fragile]
\frametitle{\textcolor{violet}{\textbackslash newtheorem} 2}
Naredba \textcolor{violet}{\textbackslash newtheorem*} služi za unos nenumeriranih okolina. 
{\color{brown}
\begin{rol}
Ovo je Rolleov teorem.
\end{rol}
}
Prethodni ispis smo dobili tako da smo u zaglavlje dokumenta stavili
\textcolor{violet}{\textbackslash newtheorem*\{rol\}\{Rolleov teorem\}}, a onda u dokument unijeli:
\begin{lstlisting}
\begin{rol}
Ovo je Rolleov teorem.
\end{rol}
\end{lstlisting}
\end{frame}

\begin{frame}
\frametitle{\textcolor{violet}{\textbackslash newtheorem} 3}
Ovako kreirane okoline primaju i opcionalni argument: Npr.
{\color{brown}
\begin{rol}[Ne baš]
Ovo je Rolleov teorem.
\end{rol}
}
 je dobijeno tako da smo umjesto \textcolor{violet}{\textbackslash begin\{rol\}} stavili 
\textcolor{violet}{\textbackslash begin\{rol\}[Ne baš]}.

Predefinirano ponašanje ovih okolina je da svaka od njih ima zasebnu numeraciju.
\end{frame}

\begin{frame}[fragile]
\frametitle{Tipovi okolina}
Ukoliko želimo da npr. korolari dijele numeraciju zajedno s teoremima definiramo
\begin{lstlisting}
\newtheorem{cor}[thm]{Korolar}
\end{lstlisting}
Efekt je 
\textcolor{violet}{
\begin{cor}
Evo prvog korolara.
\end{cor}
}
Postoje tri osnovna tipa okolina: \textcolor{violet}{plain} (predefinirana), \textcolor{violet}{definition} i \textcolor{violet}{remark}.
Naravno, možemo i sami definirati izgled, ukoliko nam ova tri tipa nisu dovoljna. Npr. ukoliko želimo da slutnje budu tipa \textcolor{violet}{remark} a definicije tipa \textcolor{violet}{definition}, k\^od je:
\begin{lstlisting}
\theoremstyle{remark}
\newtheorem{slutnja}{Slutnja}
\theoremstyle{definition}
\newtheorem{dfn}[slutnja]{Definicija}
\end{lstlisting}
\end{frame}

\begin{frame}
\frametitle{Tipovi okolina}
Primjena:
{\color{brown}
\begin{slutnja}
Slutnja.
\end{slutnja}
\begin{dfn}
Definicija.
\end{dfn}
}    
Postoje još mnoge naredbe u paketu \textcolor{violet}{amsthm}. Npr. ukoliko želimo da numeracija dolazi ispred imena, dovoljno je u zaglavlje (prije definicije okoline) staviti naredbu \textcolor{violet}{\textbackslash swapnumbers}.
\end{frame}

\begin{frame}[fragile]
\frametitle{\textcolor{violet}{proof}}
Također, paket \textcolor{violet}{amsthm} definira okolinu \textcolor{violet}{proof} za unos dokaza:
\begin{proof}
Dokaz.
\end{proof}
\begin{lstlisting}
\begin{proof}
Dokaz.
\end{proof}
\end{lstlisting}
\begin{block}{}
Piše \textcolor{brown}{dokaz} jer imamo uključen paket \textcolor{violet}{babel} s opcijom \textcolor{violet}{croatian}!
\end{block}
\end{frame}

\begin{frame}
\frametitle{Referenciranje}
Ove okoline (kao i sve druge) možemo referencirati pomoću naredbe \textcolor{violet}{\textbackslash label}, koju stavljamo neposredno poslije \textcolor{violet}{\textbackslash begin\{...\}}. 

Naredbu \textcolor{violet}{\textbackslash label} možemo koristiti i drugdje. Obično je koristimo da označimo sekcije, poglavlja i drugih cjelina, no možemo je koristiti i npr. kod lista. Primjer: ukoliko stavimo
\textcolor{violet}{\textbackslash section\{Uvod\} \textbackslash label\{sek:uvod\}},  u nastavku teksta možemo reći: \textcolor{violet}{kao što smo spomenuli u sekciji  \textbackslash ref\{sek:uvod\}}. Ili
\begin{enumerate}
\item Svi ljudi su smrtni 
\item Sokrat je čovjek 
\item Dakle, Sokrat je smrtan 
\end{enumerate}
1 i 2 povlači 3.
\end{frame}

\begin{frame}[fragile]
\frametitle{Referenciranje}
Prethodni tekst je dobiven pomoću sljedećeg k\^oda:
\begin{lstlisting}
\begin{enumerate}
\item Svi ljudi su smrtni \label{m:1}
\item Sokrat je čovjek \label{m:2}
\item Dakle, Sokrat je smrtan\label{m:3}
\end{enumerate}
\ref{m:1} i \ref{m:2} povlači \ref{m:3}.
\end{lstlisting}
Na taj način npr.\ ubacivanje novog poglavlja usred knjige ne predstavlja problem za referenciranje. 

Katkada se želimo referencirati na stranicu na kojoj smo npr. uveli neki pojam. U tom slučaju koristimo naredbu \textcolor{violet}{\textbackslash pageref\{oznaka\}}, ukoliko smo uz naš pojam stavili naredbu 
\textcolor{violet}{\textbackslash label\{oznaka\}}. 

Ukoliko želimo da numeracija prati npr. sekcije, u zaglavlje stavimo \textcolor{violet}{\textbackslash numberwithin\{equation\}\{section\}}
\end{frame}

\begin{frame}[fragile]
\frametitle{Numeracija: \textcolor{violet}{subequations}}
    
Ukoliko želimo nizu formula dati zajedničku oznaku, možemo koristiti okolinu \textcolor{violet}{subequations}:
\begin{subequations}
\label{sustav}
\begin{align}
2x+3y&=7 \label{s1}\\
3x-4y&=11 \label{s2}
\end{align}
\end{subequations}
Ovaj prikaz smo dobili pomoću sljedećeg k\^oda:

\begin{lstlisting}
\begin{subequations}
\label{sustav}
\begin{align}
2x+3y&=7 \label{s1}\\
3x-4y&=11 \label{s2}
\end{align}
\end{subequations}
\end{lstlisting}

\end{frame}

\end{document}
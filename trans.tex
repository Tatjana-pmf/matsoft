\documentclass[10pt]{beamer}
\mode<presentation>{%
	\usetheme{Goettingen}
%	\usetheme{Warsaw}
%	\usetheme{Frankfurt}
%	   \usecolortheme{seahorse}
%	   \usecolortheme{rose}
%	   \usetheme{Berkeley}
}


\title{Simple Presentation}
\author[D. Flipo, Euclid]{Daniel Flipo \& Euclid}
\institute{U.S.T.L. \& GUTenberg}
\date{2005}

\begin{document}
	%
	\begin{frame}<hamndout:0>
		\titlepage		
	\end{frame}

	\begin{frame}
%	\transdissolve
	\transdissolve[duration=3]

	\frametitle{Outline}
	\tableofcontents
%	\tableofcontents[pausesections]
	\end{frame}

	\section{An example}

		\begin{frame}
%			\transboxout
			\transboxin
%			\transsplitverticalout
			\frametitle{Things to do on a Sunday Afternoon}
			\begin{block}{One could \ldots}
				\begin{itemize}
					\item walk the dog \dots\pause
					\item read a book\dots\pause
					\item confuse a cat\dots\pause
					\item \fbox{solve $y^\prime(x)=\max\{x,y(x)\}$}!
%					\item solve 
%					\[
%					\fbox
%					{$y^\prime(x)=\max\{x,y(x)\}
%					 $}!
%					%}
%					\]
				\end{itemize}
			\end{block}
		\end{frame}
		
	\section{Another example}
				
		\subsection{Our motivation}
			%
		\begin{frame}
				   \transwipe[direction=90]
				   \frametitle{What Are Prime Numbers?}
			\begin{definition}
				A \alert{prime number} is a number that 
				has exactly two divisors.
			\end{definition}
			\begin{example}
					%
					\begin{itemize}
						\item 2 is prime (two divisors: 1 and 2).
						\item 3 is prime (two divisors: 1 and 3).
						\item 4 is not prime (\alert{three} divisors: 
								1, 2, and 4).
					\end{itemize}
					%
			\end{example}
			%
		\end{frame}
		
		\subsection{Our Theorem}
			%
		\begin{frame}
				   \transglitter<2-3>[direction=90]
			\frametitle{There Is No Largest Prime Number}
			\framesubtitle{The proof uses \textit{reductio ad absurdum}.}
			%
			\begin{theorem}
					There is no largest prime number.
			\end{theorem}
			%
			\begin{proof}
				%
				\begin{enumerate}
					\item<1-| alert@1> Suppose $p$ were the largest
							 prime number.
					\item<2-> Let $q$ be the product of the first $p$ 
							 	numbers.
					\item<3-> Then $q+1$ is not divisible by any of them.
					\item<4-> Thus $q+1$ is also prime and greater than 
								$p$.\qedhere
				\end{enumerate}
				%
				\uncover<4->{The proof used 
				\textit{reductio ad absurdum}.}	
			\end{proof}
			%
		\end{frame}
		
		
\end{document}  
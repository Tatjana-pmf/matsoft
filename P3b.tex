\documentclass{beamer}
\usetheme{CambridgeUS}

\usepackage[croatian]{babel}
\usepackage[utf8]{inputenc}
\usepackage[T1]{fontenc}
\usepackage[croatian]{csquotes}
\usepackage{sagetex}
\usepackage{amsmath}
\usepackage{url}
\usepackage{fancybox}
\usepackage{wasysym}
\usepackage{listings}
\usepackage{verbatim}
\usepackage{multirow}
\usepackage{graphicx}

\newcommand{\A}[1]{#1&\texttt{\string#1}\hspace*{1ex}}


\hypersetup{plainpages=false,bookmarksopen,linkcolor=violet,%
pdfpagemode=FullScreen,colorlinks=true%
}
\hypersetup{pdftitle={Prezentacije}}
\hypersetup{baseurl={http://www.math.hr}}
\hypersetup{pdfsubject={LaTeX}}
\hypersetup{pdfauthor={Ivica Nakić}}

\lstset{language=[LaTeX]TeX,extendedchars=true,backgroundcolor=\color{gray!20},identifierstyle=,stringstyle=\rmfamily,%
columns=flexible,showstringspaces=false,keywordstyle=\color{violet}\bfseries}

\lstset{literate=%
{Š}{{\v{S}}}1
{Đ}{{\Dj{}}}1
{Č}{{\v{C}}}1
{Ć}{{\'C}}1
{Ž}{{\v{Z}}}1
{š}{{\v{s}}}1
{đ}{{\dj{}}}1
{č}{{\v{c}}}1
{ć}{{\'c}}1
{ž}{{\v{z}}}1
}

\lstdefinestyle{SageInput}{style=DefaultSageInput,numbers=none}
\lstdefinestyle{SageOutput}{style=DefaultSageOutput,numbers=none}

\title{\LaTeX{} --- Sage i Sage\TeX}
\subtitle{}
\author{Ivica Nakić \\ \texttt{nakic@math.hr}}
\institute[PMF--MO]{Matematički odsjek Prirodoslovno--matematičkog fakulteta}
\date[2014/15]{Matematički software, 2014/15}


\begin{document}

\begin{frame}
  \maketitle  
\end{frame}

\begin{frame}
\frametitle{Pregled}
  \tableofcontents  
\end{frame}

\section{Sage}

\begin{frame}
\frametitle{Što je Sage?}
\href{http://www.sagemath.org}{Sage} je matematička aplikacija sa sučeljem u programskom jeziku Python. Sage u sebi sadrži cijeli niz biblioteka (napisanih u raznim programskim jezicima) za razna područja matematike. Python biblioteke koje smo dosad obradili (Numpy, Scipy, Matplotlib, Sympy, Cython) su također ugrađene, ali unutar Sagea imamo pristup i R-u, Fortranu, C-u, C++-u,\ldots   
\pause

Sadrži biblioteke za algebru, algebarsku geometriju, aritmetičku geometriju, diferencijalni i integralni račun, kombinatoriku, linearnu algebru, teoriju grafova, teoriju grupa, teoriju brojeva,\ldots
\pause

Sage je nastao 2005.\ godine, trenutno je u 6.6 inačici te je dostupan za Unix operativne sustave te online na Sagemath oblaku.
\end{frame}

\begin{frame}
\frametitle{Python vs.\ Sage}
Iako u Sageu programiramo u Pythonu, postoje dvije razlike koje treba držati na umu:
\begin{itemize}
     \item potenciranje se može pisati i pomoću simbola \textasciicircum{} (xor operacija se piše s \textasciicircum\textasciicircum),
     \item izrazi oblika $2/3$ ne vraćaju nulu već $2/3$ (dakle, razlomak); ako želimo cjelobrojno dijeljenje pišemo $2//3$.
   \end{itemize}   
Ove razlike su dobijene korištenje pre-parsera.

\end{frame}

\begin{frame}[fragile]
\frametitle{Sage kao kalkulator}
Neki jednostavni primjeri koda u Sageu:

\begin{sageblock}
4 * (10 // 4) + 10 % 4 == 10
3^2*4 + 2%5
sqrt(3.4)
sin(pi/3)
exp(2)
sin(10).n(digits=5)
N(sin(10),digits=10)
sqrt(pi).numerical_approx()
numerical_approx(pi, prec=200)
\end{sageblock}

\end{frame}

\begin{frame}[fragile]
\frametitle{Neki tipovi varijabli su promijenjeni}
\begin{sagecommandline}
    sage: a = 5
    sage: type(a)
\end{sagecommandline}
\begin{sagecommandline}
    sage: a = 5/3
    sage: type(a)
\end{sagecommandline}
\begin{sagecommandline}
    sage: a = 'hello'
    sage: type(a)
\end{sagecommandline}      
\end{frame}

\begin{frame}[fragile]\frametitle{Rješavanje jednadžbi}
Ako izvršimo sljedeći kod (slično kao i u Sympy-ju)    
\begin{sageblock}
    x = var('x')
    solve(x^2 + 3*x + 2, x)
\end{sageblock}  
za rješenje ćemo dobiti $\sage{solve(x^2 + 3*x + 2, x)}$. 

Analogno,
\begin{sageblock}
    x, b, c = var('x b c')
    solve([x^2 + b*x + c == 0],x)  
\end{sageblock}
nam daje $\sage{solve([x^2 + b*x + c == 0],x)}$. 
\end{frame}

\begin{frame}[fragile]\frametitle{Rješavanje jednadžbi}
Sustav nelinearnih jednadžbi:    
\begin{sageblock}
  var('x y p q')
  eq1 = p+q==9
  eq2 = q*y+p*x==-6
  eq3 = q*y^2+p*x^2==24
  solve([eq1,eq2,eq3,p==1],p,q,x,y)
\end{sageblock} 
\begin{sagesilent}
  s = solve([eq1,eq2,eq3,p==1],p,q,x,y)
\end{sagesilent}
Rješenja su \[\sage{s[0]},\] \[\sage{s[1]}.\]  
\end{frame}

\begin{frame}[fragile]\frametitle{Rješavanje jednadžbi}
Numeričko nalaženje rješenja:
\begin{sagecommandline}
    sage: phi = var('phi')
    sage: find_root(cos(phi)==sin(phi),0,pi/2)
\end{sagecommandline}
Rješavanje nejednadžbi:
\begin{sagecommandline}
    sage: solve(x^2+x-1 > 0, x)
\end{sagecommandline}
\end{frame}

\begin{frame}[fragile]\frametitle{Nalaženje korijena polinoma}
\begin{sageverbatim}
    sage: (x^3+2*x+1).roots(x)
\end{sageverbatim}
\begin{sagesilent}
    s = (x^3+2*x+1).roots(x)
\end{sagesilent}
Korijeni su 
\begin{gather*}
\sage{s[0]},\\ \sage{s[1]},\\ \sage{s[2]}. 
\end{gather*}    
\end{frame}


\begin{frame}[fragile]\frametitle{Nalaženje korijena polinoma II}
\begin{sageverbatim}
    sage: (x^3+2*x+1).roots(x, ring=RR)
\end{sageverbatim}
daje $\sage{(x^3+2*x+1).roots(x, ring=RR)}$.
\begin{sageverbatim}
    sage: (x^3+2*x+1).roots(x, ring=CC)
\end{sageverbatim}
\begin{sagesilent}
    s = (x^3+2*x+1).roots(x, ring=CC)
\end{sagesilent}
daje 
$\sage{s[0]}$, $\sage{s[1]}$, $\sage{s[2]}$.
\end{frame}

\begin{frame}[fragile]
\frametitle{Deriviranje}
\begin{sagecommandline}
    sage: diff(sin(x^2), x, 4)
\end{sagecommandline}
\begin{sagecommandline}
    sage: x, y = var('x,y')
    sage: f = x^2 + 17*y^2
    sage: f.diff(y)
\end{sagecommandline}
\end{frame}

\begin{frame}[fragile]\frametitle{Integriranje, parcijalni razlomci}
\begin{sagecommandline}
    sage: integral(x*sin(x^2), x)
    sage: integral(x/(x^2+1), x, 0, 1)
\end{sagecommandline}    
Za numeričko integriranje postoji funkcija \texttt{integral\_numerical}.

\begin{sagecommandline}
  sage: f = 1/((1+x)*(x-1))
  sage: f.partial_fraction(x)
\end{sagecommandline}  

Ako nemamo problema s automatskim kreiranjem varijabli možemo koristiti:
\begin{sageverbatim}
automatic_names(True)  
\end{sageverbatim}
\end{frame}

\begin{frame}[fragile]\frametitle{Pojednostavljivanje}
\begin{sagecommandline}    
  sage: simplify(arccos(sin(pi/3)))
  sage: simplify(exp(i*pi/6))
\end{sagecommandline}    
\begin{sagecommandline}
  sage: a = var('a')  
  sage: y = cos(x+a) * (x+1)
  sage: y.subs(a=-x)
  sage: y.subs(x=pi/2, a=pi/3)
\end{sagecommandline}    
\end{frame}
\begin{frame}[fragile]
\frametitle{Pojednostavljivanje, simboličke funkcije}
\begin{sagecommandline}    
  sage: y, z = var('y, z')
  sage: f = x^3 + y^2 + z
  sage: f.subs_expr(x^3 == y^2, z==1)
\end{sagecommandline}  
\begin{sagecommandline}    
  sage: f(x)=(2*x+1)^3
  sage: f(-3)
  sage: f.expand()
\end{sagecommandline}    
\begin{sagecommandline} 
  sage: ((x+y+sin(x))^2).expand().collect(sin(x))
\end{sagecommandline}   
\end{frame} 

\begin{frame}[fragile]\frametitle{Pojednostavljivanje, simboličke funkcije}
\begin{sagecommandline}    
  sage: u = sin(x) + x*cos(y)
  sage: v = u.function(x, y)
  sage: v
\end{sagecommandline}    
\begin{sagecommandline}    
  sage: f = (e^x-1) / (1+e^(x/2))
  sage: f.simplify_exp()
\end{sagecommandline} 
\begin{sagecommandline}    
  sage: f = cos(x)^6 + sin(x)^6 + 3 * sin(x)^2 * cos(x)^2 
  sage: f.simplify_trig()
\end{sagecommandline} 
\end{frame}

\begin{frame}[fragile]\frametitle{Pojednostavljivanje}
\begin{sagecommandline}    
  sage: f = cos(x)^6
  sage: f.reduce_trig() 
  sage: f = sin(5 * x)
  sage: f.expand_trig()   
  sage: n = var('n')
  sage: f = factorial(n+1)/factorial(n)
  sage: f.simplify_factorial()
  sage: f = sqrt(abs(x)^2) 
  sage: f.simplify_radical()
\end{sagecommandline} 
Ostale naredbe: \texttt{factor}, \texttt{factor\_list}, \texttt{combine}, \texttt{simplify\_full},\ldots
\end{frame}

\begin{frame}[fragile]\frametitle{Uvjeti na varijable}
\begin{sagecommandline}    
  sage: assume(x > 0)
  sage: bool(sqrt(x^2) == x) 
  sage: forget(x > 0)
  sage: bool(sqrt(x^2) == x)
  sage: n = var('n')
  sage: assume(n, 'integer')
  sage: sin(n*pi).simplify()
\end{sagecommandline}     


\end{frame}
 
\begin{frame}[fragile]\frametitle{Rješavanje ODJ}
  
\begin{sagecommandline}
  sage: t = var('t')
  sage: x = function('x',t)
  sage: DE = diff(x, t) + x - 1
  sage: desolve(DE, [x,t])
\end{sagecommandline} 
\begin{sagecommandline}
  sage: x = var('x')
  sage: y = function('y', x)
  sage: desolve(diff(y,x,x) + x*diff(y,x) + y == 0, y, [0,0,1])
\end{sagecommandline} 
\end{frame} 

\begin{frame}[fragile]\frametitle{Sume, redovi}
\begin{sagecommandline}
  sage: k, n = var('k, n')
  sage: sum(k, k, 1, n).factor()
  sage: sum(k * binomial(n, k), k, 0, n)
  sage: n, k, y = var('n, k, y')
  sage: sum(binomial(n,k) * x^k * y^(n-k), k, 0, n)
  sage: a, q, k, n = var('a, q, k, n')
  sage: sum(a*q^k, k, 0, n)
\end{sagecommandline}     
\end{frame}

\begin{frame}[fragile]\frametitle{Sume, redovi}
\begin{sagecommandline}
  sage: a, q, k, n = var('a, q, k, n')
  sage: sum(a*q^k, k, 0, n)
  sage: assume(abs(q) < 1)
  sage: sum(a*q^k, k, 0, infinity)
\end{sagecommandline}     
\end{frame}

\begin{frame}[fragile]\frametitle{Limesi}
\begin{sagecommandline}
  sage: limit((x**(1/3) - 2) / ((x + 19)**(1/3) - 3), x = 8)
  sage: f(x) = (cos(pi/4-x)-tan(x))/(1-sin(pi/4 + x))
  sage: limit(f(x), x = pi/4)
  sage: limit(f(x), x = pi/4, dir='minus')
  sage: limit(f(x), x = pi/4, dir='plus')
  sage: u(n) = n^100 / 100^n
  sage: limit(u(n), n=infinity)
\end{sagecommandline}  

\end{frame}

\begin{frame}[fragile]\frametitle{Razvoj funkcije u red}
\begin{sagecommandline}
  sage: taylor((1+arctan(x))**(1/x), x, 0, 3)
  sage: (ln(2*sin(x))).series(x==pi/6, 3)
  sage: (ln(2*sin(x))).series(x==pi/6, 3).truncate()
  sage: f = arctan(x).series(x, 10)
  sage: f
  sage: (16*f.subs(x==1/5) - 4*f.subs(x==1/239)).n()
\end{sagecommandline}   
\end{frame}

\section{Sage\TeX}

\begin{frame}\frametitle{Sage\TeX}
Ovaj \LaTeX{} dokument je kreiran uz pomoć paketa Sage\TeX{} koji omogućava da u jednom dokumentu imamo k\^od i za \LaTeX{} i za Sage.

Korištenje ovog paketa je automatizirano na Sagemath oblku, za korištenje na vlastitom računalu svakako je potrebno imati i \TeX{} distribuciju i Sage.

Naredbe koje se koriste za ispisivanje i procesiranje Sage k\^oda su:

\begin{itemize}
  \item \textcolor{violet}{\textbackslash sage}
  \item \textcolor{violet}{\textbackslash sageplot}
  \item \textcolor{violet}{\textbackslash sagestr}
  \item \textcolor{violet}{\textbackslash percent}
\end{itemize}

Okoline koje se koriste za ispisivanje i procesiranje Sage k\^oda su:
\begin{itemize}
  \item \textcolor{violet}{sageblock}
  \item \textcolor{violet}{sagesilent}
  \item \textcolor{violet}{sageverbatim}
  \item \textcolor{violet}{comment}
  \item \textcolor{violet}{sagecommandline}
\end{itemize}
\end{frame}

\begin{frame}[fragile]\frametitle{\textbackslash sage, sagestr, percent}
Rezultat naredbe
\begin{lstlisting}
$\sage{matrix([[1, 2], [3,4]])^2}$
\end{lstlisting}  
je
\[\sage{matrix([[1, 2], [3,4]])^2}\]

Naredba \textcolor{violet}{\textbackslash sagestr} je analogna naredbi \textcolor{violet}{\textbackslash sage}, samo što ne vraća \LaTeX{} k\^od.

Naredba \textcolor{violet}{\textbackslash percent} služi za unošenje znaka \% u naredbe \textcolor{violet}{\textbackslash sage} i \textcolor{violet}{\textbackslash sageplot}.
\end{frame}

\begin{frame}[fragile]\frametitle{\textbackslash sageplot}
Naredba \textcolor{violet}{\textbackslash sageplot} služi za ubacivanje slika:
\begin{lstlisting}
\sageplot[angle=30, width=8cm][png]{plot(sin(x), 0, pi), axes=False}  
\end{lstlisting}
\begin{center}
\sageplot[angle=30, width=8cm][png]{plot(sin(x), 0, pi), axes=False}    
\end{center}
\end{frame}

\begin{frame}[fragile]\frametitle{sageblock}
Tekst unutar okoline \textcolor{violet}{sageblock} će biti ispisan te izvršen. Npr.\ k\^od
\begin{lstlisting}
\begin{sageblock}
        var('x')
        f(x) = sin(x) - 1
        g(x) = log(x)
        h(x) = diff(f(x) * g(x), x)
\end{sageblock}  
\end{lstlisting}
će biti ispisan kao
\begin{sageblock}
        var('x')
        f(x) = sin(x) - 1
        g(x) = log(x)
        h(x) = diff(f(x) * g(x), x)
\end{sageblock}
\end{frame}

\begin{frame}[fragile]\frametitle{sageblock, sagesilent, sageverbatim, comment}
Nadalje, sljedeći će k\^od    
\begin{lstlisting}
Vrijedi $h(2) = \sage{h(2)}$, gdje je $h$ derivacija produkta 
funkcija $f$ i $g$.  
\end{lstlisting}
biti ispisan kao
\begin{block}{}
Vrijedi $h(2) = \sage{h(2)}$, gdje je $h$ derivacija produkta funkcija $f$ i $g$.  
\end{block}
Okolina \textcolor{violet}{sagesilent} radi isto što i \textcolor{violet}{sageblock}, ali ne ispisuje k\^od. \textcolor{violet}{sageverbatim} ispisuje k\^od ali ga ne izvršava. \textcolor{violet}{comment} niti ispisuje niti izvršava k\^od te služi za upisivanje Sage komentara u \LaTeX{} k\^od.
\end{frame}

\begin{frame}[fragile]\frametitle{sagecommandline}
Ovu naredbu ćemo ilustrirati primjerom:
Sljedeći k\^od 
\begin{lstlisting}
\begin{sagecommandline}
       sage: 1+1
       sage: factor(x^2 + 2*x + 1)
\end{sagecommandline}        
\end{lstlisting}    
ispisuje:
\begin{sagecommandline}
       sage: 1+1
       sage: factor(x^2 + 2*x + 1)
\end{sagecommandline} 
\end{frame}
\end{document}